%%%%%%%%%%%%%%%%%%%%%%%%%%%%%%%%%%%%%%%%%%%%%%%%%%%%%%%%%%%%%%%%%%%%%%%%%%%%%%%%%%%%%%%%%%%%%%%%%%%%%%%%%

%% Modelo desenvolvido por Alunos do IFSP - PEP para o CONICT 2020.
% Autores: Christopher Alves de Oliveira, Inácio Ribeiro, Paulo Ricardo Servilha Sparapan.
% Professores: Alexandre Ataide Carniato e Diego Nunes da Silva
% Data: 18/08/2020.

%%%%%%%%%%%%%%%%%%%%%%%%%%%%%%%%%%%%%%%%%%%%%%%%%%%%%%%%%%%%%%%%%%%%%%%%%%%%%%%%%%%%%%%%%%%%%%%%%%%%%%%%%

% Atualizado para o CONICT 2021 - Cubatão pelo professor Glauber Renato Colnago 
% Data: 7/6/2021
% Atualizado para o CONICT 2022 - São Paulo pelo professor Francisco Yastami Nakamoto 
% Data: 24/06/2022
% Atualizado para o CONICT 2023 - Capivari pelo professor Thiago Pedro Donadon Homem
% Data: 24/05/2023
% Atualizado para o CONICT 2024 - Barretos pelo professor Rodrigo Yamakami Camilo
% Data: 09/05/2024
%%%%%%%%%%%%%%%%%%%%%%%%%%%%%%%%%%%%%%%%%%%%%%%%%%%%%%%%%%%%%%%%%%%%%%%%%%%%%%%%%%%%%%%%%%%%%%%%%%%%%%%%%

\documentclass[a4paper,11pt]{article} %Selecionando a classe que gera artigos.

%PACOTES ABNT
\usepackage[brazil]{babel} % Pacote para usar portugues.
%Babel traduz como “table of contents“, “chapter” e “section“, sejam impressas como “sumário”, “capítulo” e “seção”, respectivamente.

\usepackage[T1]{fontenc} %Pacote para copiar palavras com acentuação do pdf de um documento feito em LaTeX que tenha usado esses pacotes resultará nas palavras corretamente acentuadas

\usepackage[utf8]{inputenc} % Pacote para poder usar acentuação no arquivo .tex

%\usepackage{biblatex}


\usepackage{graphicx} %Pacote para inserir Figuras.

\usepackage[top=2.0cm,right=2.0cm,left=2.0cm,bottom=2.0cm]{geometry} %Pacote de Margens

\usepackage{enumerate} %Pacote para usar marcadores.

\usepackage[onehalfspacing]{setspace} %Pacote espaçamento 1,5 cm.

\usepackage[alf]{abntex2cite} % configura o sistema  de citações e referências para o estilo ABNT.


% alf = estilo autor-data, para citações.

%Para mais estilos de citações visite https://www.overleaf.com/learn/latex/Biblatex_bibliography_styles

%PACOTES ESSENCIAIS
\usepackage{graphicx,xcolor,comment,enumerate,multirow,multicol,indentfirst} %Pacote para inserir figuras, tabelas, editar cores das tabelas, numeração "Automática das figuras e tabelas", identar tabelas e figuras.


%PACOTES DE MATEMÁTICA
\usepackage{amsmath,amsthm,amsfonts,amssymb,dsfont,mathtools,blindtext} %Pacotes básicos de matemática.

\usepackage{sectsty}% http://ctan.org/pkg/sectsty
\usepackage{titlecaps}% http://ctan.org/pkg/titlecaps
\sectionfont{\normalsize\MakeUppercase}


\usepackage{fancyhdr}  % Para adicionar o rodapé
 %Pacotes principais

\fancypagestyle{empty}{%
\fancyhf{}% clear all header and footer fields
\fancyfoot[L]{15º CONICT 2024} % except the center
\fancyfoot[C]{\thepage} % except the center
\fancyfoot[R]{ISSN: 2178-9959} % except the center
\renewcommand{\headrulewidth}{0pt}%
\renewcommand{\footrulewidth}{0pt}%
}
\pagestyle{empty}

\begin{document} %Begin Inicia o Documento

%--------------------------------     NÃO ALTERAR    ----------------------------------------%
\begin{center}

% \begin{table}[!h]
% \centering
% \resizebox{\textwidth}{!}{
% \begin{tabular}{lr}
% %\multirow{4}{*}{\includegraphics[width=80px]{logo_reitoria.png}} &
% % \multirow{6}{*}{\includegraphics[width=120px]{logo_conict.png}}                                                                     & 
% % \multirow{6}{*}{\includegraphics[width=190px]{logo_reitoria.png}} \\
% \multirow{1}{*}{
%   \centering
%   % \includegraphics[width=120px]{logo_conict.png}
%   } &
% \multirow{6}{*}{
%   \centering
%   \includegraphics[width=130px]{logo_ifsp.png}
%   } \\
%                         &                                                        %                    &            
%                         \\
%                         &                                                         %                                    &                         
%                         \\
%                         & \multicolumn{1}{l}{\textbf{}}                          %                                     &                        
%                         \\
%                         & \multicolumn{1}{l}{\textbf{}}                          %                                     &                        
%                         \\
%                         %& 
%                         \multicolumn{2}{c}{\large{\textbf{Linguagem e Descrição de Hardware}}} \\
%                         %& \multicolumn{1}{l}{}   
% \end{tabular}
% }

% \end{table}

\begin{table}[!h]
  \centering
  \resizebox{\textwidth}{!}{
  \begin{tabular}{c}
  \includegraphics[width=130px]{imgs/logo_ifsp.png} \\
  \large{\textbf{Linguagem e Descrição de Hardware}} \\
  \end{tabular}
  }
\end{table}

%-------------------------------------------------------------------------------------------------%



%%%%%%%%%%%%%%%%%%%%%%%%%%%%%  ALTERAR TÍTULO, AUTORES  %%%%%%%%%%%%%%%%%%%%%%%%%%%%%%%%%%%%%%%%%%%%%

\textbf{Greatest Common Divisor - (GCD)}\vspace{0.5cm}

Giovanna Fantacini$^1$, 
Higor Grigorio$^2$, 
Leonardo Reneres$^3$, 
Luis Boni$^4$, 
Raul Prado Dantas$^5$


%%%%%%%%%%%%%%%%%%%%%%%%%%%%%%%%%%%%%%%%%%%%%%%%%%%%%%%%%%%%%%%%%%%%%%%%%%%%%%%%%%%%%%%%%%%%%%%%%%%%%





%----------------------------------APAGAR TABELA COMEÇO--------------------------------------------%

% \begin{table}[!h]
% \centering
% \resizebox{\textwidth}{!}{%
% \begin{tabular}{|c|}
% \hline
% {\color[HTML]{FE0000} \textbf{\begin{tabular}[c]{@{}c@{}}Não informe o nome dos autores na etapa de avaliação, apenas na versão final. \\
% O resumo expandido deve ter no máximo 6 (seis) páginas, incluindo as referências. \\
% O arquivo de submissão deve estar em formato .PDF.\\
% Remova este quadro antes do envio.\end{tabular}}} \\ \hline
% \end{tabular}%
% }
% \end{table}

%---------------------------------APAGAR TABELA FIM--------------------------------------------%


\end{center}

%%%%%%%%%%%%%%%%%%%%%%%%%%%%INSERIR INFORMAÇÕES ALUNOS%%%%%%%%%%%%%%%%%%%%%%%%%%%%%%%%%%%%%%%%%
% \begingroup
%     \fontsize{9pt}{11pt}\selectfont
  
%   $^1$Graduando em Tecnologia de Análise e Desenvolvimento de Sistemas, Bolsista PIBIFSP, IFSP, Câmpus Capivari, emailautor@ifsp.edu.br. (Times New Roman, 9, Justificado)
  
%   $^2$
  
%   $^3$
  
%   $^n$

% Área de conhecimento (Tabela CNPq): 1.03.03.04-9 Sistemas de Informação. 
% \endgroup

%%%%%%%%%%%%%%%%%%%%%%%%%%%%%%%%%%%%%%%%%%%%%%%%%%%%%%%%%%%%%%%%%%%%%%%%%%%%%%%%%%%%%%%%%%%%%%%% %Alterar Título, autores, e apagar tabela !

%----------------------------------------TEXTOS-------------------------------------------------%

\vspace{0.5cm}
\noindent\textbf{RESUMO}: Este artigo apresenta a implementação do algoritmo para cálculo do Greatest Common Divisor (GCD) em hardware, utilizando uma abordagem de Control e Datapath. Serão abordadas as principais etapas de desenvolvimento, incluindo a descrição do algoritmo em linguagem C, a modelagem dos módulos de controle e datapath em Verilog, e a simulação e validação do sistema.

\vspace{0.5cm}
\noindent\textbf{PALAVRAS-CHAVE}: GCD; Greatest Common Divisor; Hardware Description Language; Control; Datapath.

\vspace{0.5cm}
 \begin{center}
 \textbf{TITLE IN ENGLISH}
 \end{center}

\noindent\textbf{ABSTRACT}: This paper presents the implementation of the Greatest Common Divisor (GCD) algorithm in hardware, using a Control and Datapath approach. The main development steps will be addressed, including the description of the algorithm in C language, the modeling of control and datapath modules in Verilog, and the system simulation and validation.

\vspace{0.5cm}
\noindent\textbf{KEYWORDS}: GCD; Greatest Common Divisor; Hardware Description Language; Control; Datapath.

\section*{Introdução}

O Greatest Common Divisor (GCD) é um problema clássico da teoria dos números, com aplicações em diversas áreas da computação. Este trabalho baseia-se nas notas de aula da disciplina CSE 141L: Introduction to Computer Architecture Lab, ministrada na Universidade da Califórnia. O objetivo é desenvolver uma implementação eficiente do GCD utilizando linguagem de descrição de hardware.

O cálculo do GCD é frequentemente utilizado em algoritmos de criptografia, compressão de dados e outros campos onde operações matemáticas eficientes são essenciais. Este projeto visa criar um design em hardware que maximize a eficiência dessas operações.

As referências devem estar citadas no trabalho conforme a sua forma de citação, como por exemplo \cite{alves}, \cite{galvani} e \cite{national_instruments} ou em Pandorfi, et al, (2007). Na seção Referências devem ser listadas em ordem alfabética \cite{pandorfi}.

\section*{Materiais e Métodos}

Os materiais e métodos utilizados no desenvolvimento da pesquisa incluem a descrição do algoritmo em C, a modelagem dos módulos de controle e datapath em Verilog, e a simulação utilizando o ambiente Intel QuestaSim.

\subsection*{Descrição do Algoritmo em C}

O algoritmo para cálculo do GCD pode ser implementado de diversas maneiras. Abaixo, apresentamos uma implementação em linguagem C:

\begin{verbatim}
int GCD(int inA, int inB) {
    int swap;
    int done = 0;
    int A = inA;
    int B = inB;
    while (!done) {
        if (A < B) {
            swap = A;
            A = B;
            B = swap;
        } else if (B != 0) {
            A = A - B;
        } else {
            done = 1;
        }
    }
    return A;
}
\end{verbatim}

\subsection*{Modelagem em Verilog}

A implementação do GCD em hardware envolve a criação de módulos para o controle e o datapath. O datapath lida com a movimentação e transformação dos dados, enquanto o módulo de controle gerencia as operações de controle.

\subsubsection*{Módulo Datapath}
\begin{verbatim}
module GCDdatapath#( parameter W=16 )
(
    input clk,
    input [W-1:0] operand_A,
    input [W-1:0] operand_B,
    output [W-1:0] result_data,
    input A_ld,
    input B_ld,
    input [1:0] A_sel,
    input B_sel,
    output B_zero,
    output A_lt_B,
    output [W-1:0] A_chk, B_chk, sub_chk, A_mux_chk, B_mux_chk
);

    wire [W-1:0] A;
    wire [W-1:0] B;
    wire [W-1:0] sub_out;
    wire [W-1:0] A_mux_out;
    wire [W-1:0] B_mux_out;

    Mux3#(W) A_mux
    (.in0 (operand_A),
    .in1 (sub_out),
    .in2 (B),
    .sel (A_sel),
    .out (A_mux_out) );

    register#(W) A_reg
    (.clk (clk),
    .d (A_mux_out),
    .en (A_ld),
    .q (A) );

    Mux2#(W) B_mux
    (.in0 (A),
    .in1 (operand_B),
    .sel (B_sel),
    .out (B_mux_out) );

    register#(W) B_reg
    (.clk (clk),
    .d (B_mux_out),
    .en (B_ld),
    .q (B) );

    assign B_zero = (B==0);
    assign A_lt_B = (A < B);
    assign sub_out = A - B;
    assign result_data = A;
    // send checking signals only for debugging purposes
    assign A_chk = A;
    assign B_chk = B;
    assign sub_chk = sub_out;
    assign A_mux_chk = A_mux_out;
    assign B_mux_chk = B_mux_out;
endmodule
\end{verbatim}

\subsubsection*{Módulo Controle}
\begin{verbatim}
module GCDcontrol(
    input input_available,
    output reg idle,
    input clk, reset,
    output reg A_ld, B_sel, B_ld,
    output reg [1:0] A_sel,
    input B_zero, A_lt_B,
    output reg result_rdy,
    input result_taken,
    output [1:0] State
);

    // States naming
    localparam WAIT = 2'd0;
    localparam CALC = 2'd1;
    localparam DONE = 2'd2;

    // Constants naming for A_mux selector
    localparam A_SEL_IN = 2'b00;
    localparam A_SEL_SUB = 2'b01;
    localparam A_SEL_B = 2'b10;
    localparam A_SEL_X = 2'b11;
    // Constants naming for B_mux selector
    localparam B_SEL_A = 1'b0;
    localparam B_SEL_IN = 1'b1;
    localparam B_SEL_X = 1'bx;

    reg [1:0] CurrentState, NextState;

    always @(posedge clk or posedge reset)  
    begin 
        if (reset) 
            CurrentState <= WAIT;
        else 
            CurrentState <= NextState; 
    end 

    always @(CurrentState)
    begin
        // default is to stay in the same state
        NextState <= CurrentState;
        case ( CurrentState )
            WAIT :
                if ( input_available )
                    NextState <= CALC;
            CALC :
                if ( B_zero )
                    NextState <= DONE;
            DONE :
                if ( result_taken )
                    NextState <= WAIT;
        endcase
    end 

    always @( * )
    begin
        // Default control signals
        A_sel <= A_SEL_X;
        A_ld <= 1'b0;
        B_sel <= B_SEL_X;
        B_ld <= 1'b0;
        idle <= 1'b0; 
        result_rdy = 1'b0;

        case ( CurrentState )
            WAIT :
                begin
                    idle <= 1'b1; 
                    if(input_available)begin
                        A_sel <= A_SEL_IN;
                        B_sel <= B_SEL_IN;
                        A_ld <= 1'b1;   
                        B_ld <= 1'b1;
                    end
                end
            CALC :
                if ( A_lt_B )begin
                    A_sel <= A_SEL_B;
                    B_sel <= B_SEL_A;
                    A_ld <= 1'b1;
                    B_ld <= 1'b1;
                end
                else if ( !B_zero )begin
                    A_sel <= A_SEL_SUB;
                    A_ld <= 1'b1;
                end
            DONE : 
                result_rdy <= 1'b1;
        endcase
    end

    assign State = CurrentState; 
endmodule
\end{verbatim}

\section*{Resultados e Discussão}

Para validar a implementação, foi realizada a simulação dos módulos utilizando o ambiente Intel QuestaSim. A Figura \ref{fig:simulacao} mostra a simulação do módulo GCD.

\begin{figure}[ht]
\centering
% \includegraphics[width=10cm,angle=0]{simulacao_gcd.png}
\caption{Simulação do módulo GCD}
\label{fig:simulacao}
\end{figure}

Os resultados da simulação confirmam que o design do GCD em hardware funciona conforme esperado. O módulo datapath executa corretamente as operações de subtração e troca, enquanto o módulo de controle gerencia os estados do sistema de maneira eficiente.

\section*{Conclusões}

Neste artigo, apresentamos a implementação do algoritmo GCD em hardware, detalhando a modelagem dos módulos de controle e datapath em Verilog. A simulação mostrou que a abordagem utilizada é eficiente e atende aos requisitos do projeto. Futuras melhorias podem incluir a otimização do datapath e a integração com outros módulos de um sistema maior.

\section*{CONTRIBUIÇÕES DOS AUTORES}
Raul Prado Dantas, Higor Grigorio dos Santos, Leonardo Reneres, Giovanna Fantacini e Luis contribuíram com a concepção e escopo do estudo. Higor e Raul procederam com a metodologia e experimentos. Leonardo, Giovanna e Luis escreveram o trabalho. Todos os autores contribuíram com a revisão do trabalho e aprovaram a versão submetida.

\section*{Agradecimentos}

Gostaríamos de agradecer ao Instituto Federal de Educação, Ciência e Tecnologia de São Paulo pelo suporte e infraestrutura fornecidos para a realização deste projeto. Agradecemos também aos professores Michael B. Taylor e Matt Devuyst por suas valiosas contribuições através das notas de aula da disciplina CSE 141L.

%Referências Bibliográficas
\bibliography{bibliografia.bib}
%\printbibliography[title={REFERÊNCIAS}]
%\printbibliography{bibliografia}

\end{document}
