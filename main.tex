\documentclass[a4paper,11pt]{article} %Selecionando a classe que gera artigos.

%PACOTES ABNT
\usepackage[brazil]{babel} % Pacote para usar portugues.
%Babel traduz como “table of contents“, “chapter” e “section“, sejam impressas como “sumário”, “capítulo” e “seção”, respectivamente.

\usepackage[T1]{fontenc} %Pacote para copiar palavras com acentuação do pdf de um documento feito em LaTeX que tenha usado esses pacotes resultará nas palavras corretamente acentuadas

\usepackage[utf8]{inputenc} % Pacote para poder usar acentuação no arquivo .tex

%\usepackage{biblatex}


\usepackage{graphicx} %Pacote para inserir Figuras.

\usepackage[top=2.0cm,right=2.0cm,left=2.0cm,bottom=2.0cm]{geometry} %Pacote de Margens

\usepackage{enumerate} %Pacote para usar marcadores.

\usepackage[onehalfspacing]{setspace} %Pacote espaçamento 1,5 cm.

\usepackage[alf]{abntex2cite} % configura o sistema  de citações e referências para o estilo ABNT.


% alf = estilo autor-data, para citações.

%Para mais estilos de citações visite https://www.overleaf.com/learn/latex/Biblatex_bibliography_styles

%PACOTES ESSENCIAIS
\usepackage{graphicx,xcolor,comment,enumerate,multirow,multicol,indentfirst} %Pacote para inserir figuras, tabelas, editar cores das tabelas, numeração "Automática das figuras e tabelas", identar tabelas e figuras.


%PACOTES DE MATEMÁTICA
\usepackage{amsmath,amsthm,amsfonts,amssymb,dsfont,mathtools,blindtext} %Pacotes básicos de matemática.

\usepackage{sectsty}% http://ctan.org/pkg/sectsty
\usepackage{titlecaps}% http://ctan.org/pkg/titlecaps
\sectionfont{\normalsize\MakeUppercase}


\usepackage{fancyhdr}  % Para adicionar o rodapé
 %Pacotes principais

\fancypagestyle{empty}{%
\fancyhf{}% clear all header and footer fields
\fancyfoot[L]{ Redes Industriais e Sistemas Supervisórios - 2024}% except the center
\fancyfoot[R]{\thepage} % except the center
% \fancyfoot[R]{ISSN: 2178-9959} % except the center
\renewcommand{\headrulewidth}{0pt}%
\renewcommand{\footrulewidth}{0pt}%
}
\pagestyle{empty}

\begin{document} %Begin Inicia o Documento

%--------------------------------     NÃO ALTERAR    ----------------------------------------%
\begin{center}

% \begin{table}[!h]
% \centering
% \resizebox{\textwidth}{!}{
% \begin{tabular}{lr}
% %\multirow{4}{*}{\includegraphics[width=80px]{logo_reitoria.png}} &
% % \multirow{6}{*}{\includegraphics[width=120px]{logo_conict.png}}                                                                     & 
% % \multirow{6}{*}{\includegraphics[width=190px]{logo_reitoria.png}} \\
% \multirow{1}{*}{
%   \centering
%   % \includegraphics[width=120px]{logo_conict.png}
%   } &
% \multirow{6}{*}{
%   \centering
%   \includegraphics[width=130px]{logo_ifsp.png}
%   } \\
%                         &                                                        %                    &            
%                         \\
%                         &                                                         %                                    &                         
%                         \\
%                         & \multicolumn{1}{l}{\textbf{}}                          %                                     &                        
%                         \\
%                         & \multicolumn{1}{l}{\textbf{}}                          %                                     &                        
%                         \\
%                         %& 
%                         \multicolumn{2}{c}{\large{\textbf{Linguagem e Descrição de Hardware}}} \\
%                         %& \multicolumn{1}{l}{}   
% \end{tabular}
% }

% \end{table}

\begin{table}[!h]
  \centering
  \resizebox{\textwidth}{!}{
  \begin{tabular}{c}
  \includegraphics[width=130px]{imgs/logo_ifsp.png} \\
  \large{\textbf{Linguagem e Descrição de Hardware}} \\
  \end{tabular}
  }
\end{table}

%-------------------------------------------------------------------------------------------------%



%%%%%%%%%%%%%%%%%%%%%%%%%%%%%  ALTERAR TÍTULO, AUTORES  %%%%%%%%%%%%%%%%%%%%%%%%%%%%%%%%%%%%%%%%%%%%%

\textbf{Greatest Common Divisor - (GCD)}\vspace{0.5cm}

Giovanna Fantacini$^1$, 
Higor Grigorio$^2$, 
Leonardo Reneres$^3$, 
Luis Boni$^4$, 
Raul Prado Dantas$^5$


%%%%%%%%%%%%%%%%%%%%%%%%%%%%%%%%%%%%%%%%%%%%%%%%%%%%%%%%%%%%%%%%%%%%%%%%%%%%%%%%%%%%%%%%%%%%%%%%%%%%%





%----------------------------------APAGAR TABELA COMEÇO--------------------------------------------%

% \begin{table}[!h]
% \centering
% \resizebox{\textwidth}{!}{%
% \begin{tabular}{|c|}
% \hline
% {\color[HTML]{FE0000} \textbf{\begin{tabular}[c]{@{}c@{}}Não informe o nome dos autores na etapa de avaliação, apenas na versão final. \\
% O resumo expandido deve ter no máximo 6 (seis) páginas, incluindo as referências. \\
% O arquivo de submissão deve estar em formato .PDF.\\
% Remova este quadro antes do envio.\end{tabular}}} \\ \hline
% \end{tabular}%
% }
% \end{table}

%---------------------------------APAGAR TABELA FIM--------------------------------------------%


\end{center}

%%%%%%%%%%%%%%%%%%%%%%%%%%%%INSERIR INFORMAÇÕES ALUNOS%%%%%%%%%%%%%%%%%%%%%%%%%%%%%%%%%%%%%%%%%
% \begingroup
%     \fontsize{9pt}{11pt}\selectfont
  
%   $^1$Graduando em Tecnologia de Análise e Desenvolvimento de Sistemas, Bolsista PIBIFSP, IFSP, Câmpus Capivari, emailautor@ifsp.edu.br. (Times New Roman, 9, Justificado)
  
%   $^2$
  
%   $^3$
  
%   $^n$

% Área de conhecimento (Tabela CNPq): 1.03.03.04-9 Sistemas de Informação. 
% \endgroup

%%%%%%%%%%%%%%%%%%%%%%%%%%%%%%%%%%%%%%%%%%%%%%%%%%%%%%%%%%%%%%%%%%%%%%%%%%%%%%%%%%%%%%%%%%%%%%%% %Alterar Título, autores, e apagar tabela !

\begin{center}
    \tableofcontents %Gera o sumário
\end{center}

\newpage

%----------------------------------------TEXTOS-------------------------------------------------%

\vspace{0.5cm}
\noindent\textbf{RESUMO}:
Este artigo apresenta uma análise detalhada dos principais protocolos de comunicação utilizados em redes industriais, abordando aspectos como funcionamento, meio físico, vantagens, desvantagens e empresas fornecedoras. Entre os protocolos discutidos estão DeviceNet, Modbus, Profibus, Profinet, CAN, HART, OPC/DCOM, AS-i e Ethernet/IP. O estudo destaca a importância desses protocolos na automação industrial, proporcionando uma visão abrangente das suas aplicações e das considerações a serem feitas ao selecionar o protocolo adequado para diferentes cenários.

\vspace{0.5cm}
\noindent\textbf{PALAVRAS-CHAVE}:
Redes Industriais; Automação; Protocolos de Comunicação; DeviceNet; Modbus; Profibus; Profinet; CAN; HART; OPC; Ethernet/IP.

\vspace{0.5cm}
\begin{center}
    \textbf{Resumo de Tecnologias mais utilizadas em Redes Industriais}
\end{center}

\noindent\textbf{ABSTRACT}:
This paper presents a detailed analysis of the main communication protocols used in industrial networks, addressing aspects such as operation, physical medium, advantages, disadvantages, and supplier companies. The protocols discussed include DeviceNet, Modbus, Profibus, Profinet, CAN, HART, OPC/DCOM, AS-i, and Ethernet/IP. The study highlights the importance of these protocols in industrial automation, providing a comprehensive overview of their applications and the considerations to be made when selecting the appropriate protocol for different scenarios.

\vspace{0.5cm}
\noindent\textbf{KEYWORDS}:
Industrial Networks; Automation; Communication Protocols; DeviceNet; Modbus; Profibus; Profinet; CAN; HART; OPC; Ethernet/IP.



\section{Introdução}

As redes industriais desempenham um papel fundamental na automação e controle de processos em diversos setores da indústria. Com o avanço das tecnologias de automação, a integração eficiente entre dispositivos de campo, sistemas de controle e interfaces de monitoramento tornou-se crucial para garantir operações seguras, confiáveis e produtivas. A comunicação eficiente e padronizada entre esses sistemas é possibilitada por uma variedade de protocolos de comunicação industrial, cada um com suas características específicas e aplicabilidades.

Este artigo tem como objetivo fornecer uma visão geral dos principais protocolos de comunicação abordados na disciplina de Redes Industriais. Os protocolos discutidos incluem DeviceNet, Modbus, Profibus, Profinet, CAN, HART, OPC/DCOM, AS-i, e Ethernet/IP, todos amplamente utilizados em diferentes aplicações industriais.

Cada protocolo será explorado em termos de:
\begin{itemize}
    \item \textbf{Protocolo}: Descrição geral das características e especificações do protocolo.
    \item \textbf{Histórico}: Contextualização histórica, incluindo o desenvolvimento e evolução do protocolo.
    \item \textbf{Funcionamento}: Explicação detalhada de como o protocolo opera, incluindo a topologia de rede e os métodos de comunicação.
    \item \textbf{Meio Físico}: Descrição dos requisitos de cabeamento, conectores e outros aspectos físicos que suportam a comunicação.
    \item \textbf{Empresas Vendedoras}: Listagem das principais empresas que oferecem produtos e soluções baseadas no protocolo.
    \item \textbf{Vantagem}: Identificação das principais vantagens e benefícios do uso do protocolo em aplicações industriais.
    \item \textbf{Desvantagem}: Discussão sobre as limitações e desvantagens associadas ao protocolo.
\end{itemize}

Ao final, espera-se que este artigo forneça uma compreensão sólida dos diferentes protocolos de comunicação industrial, permitindo a escolha informada do protocolo mais adequado para aplicações específicas, levando em consideração fatores como custo, complexidade, desempenho e requisitos de infraestrutura.



\section{DeviceNet}

\subsection{Protocolo}
\begin{itemize}
    \item DeviceNet é um protocolo de comunicação aberto amplamente utilizado em automação industrial para troca de dados entre dispositivos de controle, como CLPs, sensores e atuadores.
    \item Classifica-se como um DeviceBus, especializado em comunicação de alta velocidade a nível de byte, tanto com equipamentos discretos quanto analógicos, oferecendo poderosas capacidades de diagnóstico.
    \item O protocolo segue o modelo de comunicação produtor/consumidor, onde qualquer nó pode iniciar um processo de transmissão, sem a necessidade de definir um mestre na rede.
\end{itemize}

\subsection{Histórico}
\begin{itemize}
    \item O DeviceNet foi desenvolvido pela Allen-Bradley (agora parte da Rockwell Automation) e posteriormente regulamentado pela ODVA (Open DeviceNet Vendors Association), uma organização global criada em 1995.
    \item Hoje, a ODVA tem mais de 300 empresas como membros, promovendo a padronização e o desenvolvimento contínuo do DeviceNet.
\end{itemize}

\subsection{Funcionamento}
\begin{itemize}
    \item O DeviceNet opera como uma rede digital multi-drop, onde cada dispositivo ou controlador atua como um nó na rede.
    \item Utiliza o protocolo CAN (Controller Area Network) para a camada de enlace de dados e o CIP (Common Industrial Protocol) para as camadas superiores da rede.
    \item A topologia pode ser em tronco/derivação (árvore) ou em cadeia linear. O mecanismo de comunicação é peer-to-peer com prioridade, herdando o esquema de arbitragem bit a bit do protocolo CAN.
    \item O DeviceNet suporta até 64 nós na rede, com mecanismos para detectar e corrigir IDs MAC duplicados.
\end{itemize}

\subsection{Meio Físico}
\begin{itemize}
    \item O meio físico do DeviceNet é composto por um cabo de par trançado que carrega tanto dados quanto energia, simplificando a instalação.
    \item A velocidade de comunicação varia de 125 Kbps a 500 Kbps, dependendo do tipo de cabo e do comprimento da rede.
    \item Resistor de terminação é necessário em ambas as extremidades do cabo para evitar reflexões de sinal, garantindo a integridade da comunicação.
\end{itemize}

\subsection{Empresas Vendedoras}
\begin{itemize}
    \item As principais empresas que oferecem produtos e soluções baseadas em DeviceNet incluem Rockwell Automation, Siemens, Omron, Schneider Electric e ABB.
\end{itemize}

\subsection{Vantagem}
\begin{itemize}
    \item O DeviceNet é amplamente aceito na indústria devido ao seu baixo custo e fácil implementação. Ele oferece uma utilização eficiente da largura de banda e permite que dispositivos sejam alimentados pelo próprio cabo que transporta o sinal.
    \item A rede é robusta e adequada para ambientes industriais exigentes, oferecendo diagnósticos avançados e suporte para uma ampla gama de dispositivos.
\end{itemize}

\subsection{Desvantagem}
\begin{itemize}
    \item As principais limitações do DeviceNet incluem a largura de banda limitada, o tamanho máximo das mensagens e o comprimento máximo do cabo, que pode restringir a escalabilidade da rede em aplicações maiores.
\end{itemize}
\section{Modbus}

\subsection{Protocolo}
\begin{itemize}
    \item O Modbus é um protocolo de comunicação aberto amplamente utilizado em automação industrial, desenvolvido originalmente pela Modicon (hoje parte da Schneider Electric) em 1979.
    \item Existem diferentes variantes do Modbus, como o \textbf{Modbus RTU} (Remote Terminal Unit), que utiliza comunicação serial, e o \textbf{Modbus TCP/IP}, que permite comunicação sobre redes Ethernet, adequado para grandes redes distribuídas.
    \item O Modbus é um protocolo mestre-escravo, onde dispositivos mestres (como CLPs ou sistemas SCADA) controlam a comunicação e os dispositivos escravos (sensores, atuadores) respondem às solicitações.
\end{itemize}

\subsection{Histórico}
\begin{itemize}
    \item O Modbus foi desenvolvido pela Modicon em 1979 como um protocolo de comunicação serial para facilitar a comunicação entre CLPs e outros dispositivos de automação industrial.
    \item Originalmente baseado em comunicação serial RS-232, o Modbus rapidamente se popularizou entre fabricantes de equipamentos, promovendo a interoperabilidade entre dispositivos de diferentes fabricantes.
    \item Com a proliferação das redes Ethernet na indústria, o Modbus evoluiu para incluir o \textbf{Modbus TCP/IP}, permitindo a comunicação em redes locais e através da internet, integrando sistemas de TI com infraestruturas industriais.
\end{itemize}

\subsection{Funcionamento}
\begin{itemize}
    \item No \textbf{Modbus RTU}, a comunicação ocorre em modo serial, utilizando interfaces RS-232 ou RS-485, com dados transmitidos em formato RTU (Remote Terminal Unit) ou ASCII.
    \item O \textbf{Modbus TCP/IP} encapsula o formato RTU ou ASCII em pacotes TCP/IP, permitindo comunicação em redes Ethernet e facilitando a integração com sistemas de TI.
    \item A comunicação segue um modelo mestre-escravo, onde o mestre envia comandos de leitura/escrita para os escravos e recebe as respostas apropriadas. O protocolo inclui mecanismos como verificação de erros (Checksum) para garantir a integridade dos dados.
\end{itemize}

\subsection{Meio Físico}
\begin{itemize}
    \item O Modbus RTU utiliza meios físicos como RS-232 ou RS-485, suportando comunicação ponto-a-ponto ou multiponto, com distâncias de até 4.000 pés (cerca de 1.200 metros) usando RS-485.
    \item O Modbus TCP/IP utiliza cabos Ethernet (par trançado), com distâncias típicas de até 100 metros entre dispositivos, suportando topologias como estrela, anel ou malha, através do uso de switches e roteadores.
    \item A taxa de dados no Modbus RTU é geralmente baixa (até 20 Kbps para RS-232 e até 100 Kbps para RS-485), enquanto no Modbus TCP/IP a taxa pode variar de 10 Mbps a 100 Gbps, dependendo da infraestrutura de rede utilizada.
\end{itemize}

\subsection{Empresas Vendedoras}
\begin{itemize}
    \item \textbf{Schneider Electric}: Oferece uma ampla gama de produtos compatíveis com Modbus, incluindo CLPs, interfaces de operador e módulos de comunicação.
    \item \textbf{Siemens}: Fornece soluções de automação, incluindo CLPs, interfaces de rede e sistemas SCADA, todos compatíveis com Modbus.
    \item \textbf{Rockwell Automation}: Através de sua marca Allen-Bradley, oferece uma variedade de produtos de automação industrial compatíveis com Modbus.
    \item \textbf{ABB}: Fornece produtos e soluções de automação industrial, como inversores de frequência, controladores e software de monitoramento.
    \item \textbf{ProSoft Technology}: Especializada em módulos de comunicação para diversos protocolos industriais, incluindo Modbus, oferecendo soluções para integração de sistemas.
    \item \textbf{Honeywell, Phoenix Contact, WAGO}: Oferecem uma variedade de dispositivos e soluções de automação compatíveis com o protocolo Modbus.
\end{itemize}

\subsection{Vantagem}
\begin{itemize}
    \item \textbf{Simplicidade}: O Modbus é um protocolo simples e fácil de implementar, tornando sua configuração e manutenção acessíveis para engenheiros e técnicos.
    \item \textbf{Compatibilidade}: É amplamente suportado pela maioria dos dispositivos industriais, permitindo fácil integração de dispositivos de diferentes fabricantes em uma rede Modbus.
    \item \textbf{Custo-efetividade}: Implementar o Modbus é geralmente mais econômico do que outros protocolos mais complexos, pois requer menos recursos de hardware e software.
    \item \textbf{Eficiência}: Proporciona comunicação eficiente em sistemas industriais, facilitando a troca rápida de dados entre dispositivos.
\end{itemize}

\subsection{Desvantagem}
\begin{itemize}
    \item \textbf{Segurança}: O Modbus original não foi projetado com segurança avançada, tornando-o potencialmente vulnerável a ataques. Versões mais recentes, como o Modbus TCP, oferecem recursos de segurança adicionais, mas ainda podem ser menos robustas do que outros protocolos modernos.
    \item \textbf{Limitações de Velocidade}: A comunicação serial do Modbus (RS-232/RS-485) tem limitações em termos de velocidade, o que pode ser problemático em sistemas que exigem transmissão rápida de dados.
    \item \textbf{Arquitetura Mestre-Escravo}: A arquitetura mestre-escravo do Modbus pode limitar a escalabilidade e a flexibilidade em alguns casos de uso, onde é necessário um controle mais distribuído.
    \item \textbf{Comprimento Máximo da Mensagem}: O tamanho máximo da mensagem no Modbus é limitado, o que pode ser uma restrição em sistemas que exigem a transferência de grandes quantidades de dados.
\end{itemize}

\section{Profibus (PA/DP/FMS)}

\subsection{Protocolo}
\begin{itemize}
    \item Profibus (Process Field Bus) é um protocolo de comunicação aberto, amplamente utilizado em automação industrial, garantindo alta interoperabilidade e independência de fornecedores.
    \item Existem três principais variantes do Profibus:
    \begin{itemize}
        \item \textbf{Profibus FMS (Fieldbus Message Specification):} Destinado a tarefas de comunicação complexas, oferecendo serviços para troca de mensagens em níveis de controle e supervisão. Foi um dos primeiros padrões estabelecidos para Profibus, mas hoje em dia é menos utilizado.
        \item \textbf{Profibus DP (Decentralized Peripherals):} Otimizado para comunicação de alta velocidade entre sistemas de automação e dispositivos periféricos descentralizados. Suporta comunicação rápida e eficiente, sendo amplamente usado em aplicações de controle em tempo real.
        \item \textbf{Profibus PA (Process Automation):} Desenvolvido para automação de processos, é usado para conectar sistemas de controle de processos com dispositivos de campo em áreas intrinsecamente seguras. Substitui sistemas convencionais de 4-20 mA.
    \end{itemize}
\end{itemize}

\subsection{Histórico}
\begin{itemize}
    \item O Profibus foi desenvolvido na Alemanha em 1987 por um consórcio de empresas e institutos de pesquisa, com o objetivo de criar um protocolo de comunicação padronizado que permitisse a interoperabilidade entre dispositivos de diferentes fabricantes.
    \item O Profibus FMS foi a primeira especificação lançada, seguida pelo Profibus DP em 1993, que oferecia uma alternativa mais rápida e simples. O Profibus PA foi introduzido em 1995, visando automação de processos.
\end{itemize}

\subsection{Funcionamento}
\begin{itemize}
    \item O Profibus se baseia no modelo OSI, utilizando as camadas física, de enlace de dados e de aplicação:
    \begin{itemize}
        \item \textbf{Camada Física:} Transporta os dados representados como um conjunto serial de bits. O meio físico típico é um cabo blindado de par trançado, mas também pode utilizar fibra óptica para maiores distâncias e imunidade a interferências eletromagnéticas.
        \item \textbf{Camada de Enlace de Dados:} Responsável pela formação de telegramas de mensagens, controle de acesso ao meio e prevenção de colisões de dados.
        \item \textbf{Camada de Aplicação:} Interface entre o usuário e o sistema, lidando com as funcionalidades de medição, controle e operação dos dispositivos.
    \end{itemize}
    \item O Profibus DP é projetado para comunicação cíclica em redes multimestre, onde dispositivos periféricos descentralizados trocam dados rapidamente com um controlador central. A comunicação acíclica é utilizada para diagnósticos e parametrização.
    \item O Profibus PA permite a transmissão de dados e alimentação de dispositivos em uma única linha de dois fios, adequado para áreas perigosas e intrinsecamente seguras.
\end{itemize}

\subsection{Meio Físico}
\begin{itemize}
    \item O meio físico para Profibus DP geralmente envolve cabos RS-485 ou fibra óptica, permitindo a comunicação em altas velocidades (até 12 Mbps).
    \item O Profibus PA utiliza cabos de par trançado, com a capacidade de operar em ambientes com exigências de segurança intrínseca. A topologia pode ser em linha, estrela ou anel.
    \item Para o Profibus FMS, o meio físico é semelhante ao DP, mas com requisitos específicos para comunicação em níveis mais altos de controle.
\end{itemize}

\subsection{Empresas Vendedoras}
\begin{itemize}
    \item Empresas que fornecem soluções baseadas em Profibus incluem Siemens, ABB, Rockwell Automation, Schneider Electric, e Endress+Hauser. Esses fornecedores oferecem dispositivos compatíveis, como CLPs, sensores, atuadores e ferramentas de diagnóstico.
\end{itemize}

\subsection{Vantagem}
\begin{itemize}
    \item Profibus é altamente confiável e oferece suporte a uma ampla gama de aplicações industriais, desde controle de processos até automação de manufatura.
    \item A interoperabilidade entre dispositivos de diferentes fabricantes e a capacidade de operar em ambientes rigorosos são grandes vantagens.
    \item O Profibus DP permite comunicação em tempo real, essencial para aplicações de controle com tempos críticos. O Profibus PA oferece maior segurança e funcionalidade em automação de processos.
\end{itemize}

\subsection{Desvantagem}
\begin{itemize}
    \item O Profibus pode ser complexo de implementar e manter, especialmente em grandes redes, devido à necessidade de configuração detalhada e gerenciamento de rede.
    \item O custo de implementação pode ser elevado, especialmente ao utilizar fibra óptica ou ao operar em ambientes intrinsecamente seguros.
    \item Profibus FMS, sendo uma tecnologia mais antiga, pode não ser ideal para novas implementações, sendo substituído pelo Profibus DP e PA em muitas aplicações.
\end{itemize}


\section{Profinet}

\subsection{Protocolo}
\begin{itemize}
    \item Profinet (Process Field Network) é um padrão de comunicação industrial desenvolvido pela PROFIBUS \& PROFINET International (PI) para a automação de processos e manufatura. 
    \item Ele integra a tecnologia Ethernet em aplicações industriais, oferecendo comunicação determinística em tempo real para controle de processos críticos.
    \item Existem três versões principais de Profinet:
    \begin{itemize}
        \item \textbf{Profinet CBA (Component-Based Automation):} Focado na comunicação entre componentes distribuídos.
        \item \textbf{Profinet IO (Input/Output):} Protocolo orientado a dispositivos para comunicação rápida entre controladores e dispositivos de campo.
        \item \textbf{Profinet IRT (Isochronous Real-Time):} Focado em comunicação de alta velocidade para controle em tempo real, particularmente em aplicações que exigem precisão no controle de movimento.
    \end{itemize}
    \item O Profinet é totalmente compatível com redes Ethernet padrão, permitindo que a automação industrial seja integrada com redes de TI e sistemas empresariais.
\end{itemize}

\subsection{Histórico}
\begin{itemize}
    \item O Profinet foi introduzido pela PI como uma evolução natural do Profibus, a fim de integrar a tecnologia Ethernet nas comunicações industriais, começando no início dos anos 2000.
    \item Foi amplamente adotado devido à crescente necessidade de integração de TI e automação, aproveitando a infraestrutura de rede Ethernet existente.
    \item O Profinet evoluiu para ser um dos protocolos mais populares no mundo da automação industrial, com suporte de uma vasta gama de fabricantes de equipamentos.
\end{itemize}

\subsection{Funcionamento}
\begin{itemize}
    \item O Profinet opera sobre Ethernet, usando o protocolo TCP/IP para comunicação não crítica e UDP/IP para comunicação em tempo real.
    \item No caso de comunicação determinística, o Profinet IRT oferece tempos de ciclo muito curtos, dividindo os ciclos de comunicação em segmentos síncronos e assíncronos. Isso garante que o tráfego crítico seja priorizado.
    \item O Profinet usa uma arquitetura modular e escalável, permitindo que os dispositivos sejam adicionados ou substituídos na rede sem interromper as operações.
    \item O Profinet IO segue o modelo de comunicação mestre-escravo, onde o mestre (controlador) se comunica com dispositivos de campo (escravos) em uma rede distribuída.
\end{itemize}

\subsection{Meio Físico}
\begin{itemize}
    \item O Profinet utiliza cabos Ethernet padrão (par trançado, Cat 5e ou superior) e conectores RJ45. Em ambientes industriais mais rigorosos, também são usados conectores M12 para garantir robustez mecânica.
    \item Para aplicações com requisitos especiais, como altas velocidades ou grandes distâncias, o Profinet pode ser implementado com fibra óptica.
    \item A infraestrutura Ethernet existente pode ser aproveitada, facilitando a integração com redes corporativas e reduzindo os custos de instalação.
\end{itemize}

\subsection{Empresas Vendedoras}
\begin{itemize}
    \item As principais empresas que oferecem produtos compatíveis com Profinet incluem Siemens, ABB, Schneider Electric, Rockwell Automation, Phoenix Contact, e Omron.
    \item Essas empresas fornecem uma ampla gama de dispositivos, como controladores lógicos programáveis (CLPs), switches Ethernet industriais, interfaces de operador, sensores, atuadores e módulos de I/O.
\end{itemize}

\subsection{Vantagem}
\begin{itemize}
    \item O Profinet oferece alta velocidade de comunicação, suportando aplicações em tempo real que requerem precisão no controle de movimento, como robótica e automação de processos.
    \item Sua compatibilidade com Ethernet permite fácil integração com redes de TI existentes, facilitando o monitoramento e a análise de dados em tempo real.
    \item A flexibilidade do Profinet o torna escalável, permitindo que redes cresçam ou sejam ajustadas sem grandes interrupções.
    \item A arquitetura modular e a capacidade de comunicação determinística garantem confiabilidade e desempenho em ambientes críticos.
\end{itemize}

\subsection{Desvantagem}
\begin{itemize}
    \item O Profinet pode exigir uma infraestrutura de rede moderna e robusta, especialmente em aplicações que exigem alta velocidade e precisão, como no caso do Profinet IRT.
    \item A configuração e o comissionamento podem ser complexos, exigindo conhecimento especializado para garantir o desempenho e a integridade da rede.
    \item Em comparação com outros protocolos, como Profibus, o custo de implementação pode ser mais elevado, especialmente quando há necessidade de adaptar uma rede Ethernet existente para um ambiente industrial.
\end{itemize}

\section{CAN (Controller Area Network)}

\subsection{Protocolo}
\begin{itemize}
    \item O CAN é um protocolo de comunicação em barramento, originalmente desenvolvido pela Bosch nos anos 1980, para ser utilizado na indústria automotiva. Permite a comunicação entre diferentes dispositivos eletrônicos como sensores, atuadores e unidades de controle, através de uma rede de comunicação digital.
    \item Atua principalmente nas camadas de Data Link e Física do modelo OSI, utilizando as técnicas CSMA/CR (Carrier Sense with Multiple Access/Collision Resolution) e AMP (Arbitration on Message Priority) para resolver colisões no barramento.
\end{itemize}

\subsection{Histórico}
\begin{itemize}
    \item O desenvolvimento do protocolo CAN começou em 1983 pela Bosch. Em 1991, a ISO publicou o primeiro padrão para o protocolo CAN (ISO 11898).
    \item Desde então, o CAN se expandiu para diversas aplicações, além da automotiva, sendo amplamente utilizado em setores como automação industrial e dispositivos médicos.
\end{itemize}

\subsection{Funcionamento}
\begin{itemize}
    \item O CAN permite que todas as unidades de controle eletrônico (ECUs) de um sistema enviem e recebam mensagens através de um barramento comum.
    \item As mensagens são transmitidas em forma de quadros, que contêm informações como o identificador do dispositivo, o tipo de dado e os dados em si.
    \item Utiliza um método de arbitragem de bits onde o bit dominante (0) tem prioridade sobre o bit recessivo (1), permitindo a resolução eficiente de colisões.
\end{itemize}

\subsection{Meio Físico}
\begin{itemize}
    \item O meio físico do protocolo CAN pode ser implementado usando diferentes tipos de cabos, como par trançado, cabo coaxial ou fibra óptica. A escolha do meio físico depende dos requisitos de desempenho, custo e segurança da aplicação.
    \item O cabo de par trançado é o mais comum devido ao seu baixo custo e facilidade de instalação, enquanto o cabo coaxial oferece maior resistência à interferência eletromagnética e a fibra óptica é imune a essas interferências e oferece maior velocidade de transmissão.
\end{itemize}

\subsection{Empresas Vendedoras}
\begin{itemize}
    \item Várias empresas fornecem sistemas CAN, incluindo hardware e software para desenvolvimento e implementação de redes CAN, como Bosch, Microchip Technology, NXP Semiconductors, Texas Instruments, Infineon Technologies e STMicroelectronics.
\end{itemize}

\subsection{Vantagem}
\begin{itemize}
    \item O protocolo CAN oferece comunicação confiável mesmo em ambientes com ruído eletromagnético, é altamente escalável e relativamente barato para implementar.
\end{itemize}

\subsection{Desvantagem}
\begin{itemize}
    \item Apesar de suas vantagens, o CAN apresenta limitações em segurança, complexidade na implementação e desempenho limitado em termos de velocidade de transmissão.
\end{itemize}

\newpage

\section{HART (Highway Addressable Remote Transducer)}

\subsection{Protocolo}
\begin{itemize}
    \item O HART é um protocolo de comunicação digital híbrido, combinando sinais analógicos e digitais, amplamente utilizado na automação industrial. Opera na camada de aplicação (Layer 7) do modelo OSI.
\end{itemize}

\subsection{Histórico}
\begin{itemize}
    \item Desenvolvido pela Rosemount Inc., uma divisão da Emerson Electric Company, na década de 1980, o protocolo HART foi lançado oficialmente em 1986. Ao longo dos anos, evoluiu com a introdução de novas versões como o HART 5, 6 e 7, este último introduzindo a comunicação sem fio (WirelessHART).
\end{itemize}

\subsection{Funcionamento}
\begin{itemize}
    \item O HART permite comunicação bidirecional sobre o mesmo par de fios usados para sinais analógicos de 4-20 mA. Utiliza modulação FSK (Frequency Shift Keying) para sobrepor o sinal digital ao analógico sem interferir nele.
    \item Opera em dois modos principais: Ponto-a-Ponto e Multidrop. No modo Ponto-a-Ponto, o sinal analógico é transmitido junto com o digital, enquanto no modo Multidrop, o sinal analógico é fixado em 4 mA, permitindo que vários dispositivos compartilhem o mesmo par de fios.
\end{itemize}

\subsection{Meio Físico}
\begin{itemize}
    \item O HART utiliza um único par de fios para comunicação, compatível com a maioria dos sistemas de automação industrial. O meio físico é tipicamente um cabo de par trançado, que transporta o sinal analógico de 4-20 mA junto com o sinal digital FSK.
\end{itemize}

\subsection{Empresas Vendedoras}
\begin{itemize}
    \item O HART é amplamente suportado por empresas como Emerson, ABB, Siemens, Honeywell, Yokogawa, Endress+Hauser, entre outras.
\end{itemize}

\subsection{Vantagem}
\begin{itemize}
    \item O HART é custo-efetivo, simples de implementar, e oferece flexibilidade ao ser compatível com uma ampla gama de dispositivos de campo e sistemas de controle. Também permite a manutenção preditiva e diagnósticos remotos.
\end{itemize}

\subsection{Desvantagem}
\begin{itemize}
    \item Limitações incluem largura de banda e distância limitadas, além de ser um protocolo de comunicação um-para-um, o que pode ser uma desvantagem em cenários com muitos dispositivos.
\end{itemize}


\section{OPC/DCOM}

\subsection{Protocolo}
\begin{itemize}
    \item O protocolo OPC/DCOM (OLE for Process Control/Distributed Component Object Model) é uma tecnologia fundamental para a comunicação entre sistemas de controle e automação em ambientes industriais.
    \item O OPC permite a troca de dados em tempo real entre diferentes dispositivos e softwares, como sensores, controladores, e sistemas SCADA, facilitando a interoperabilidade em ambientes de automação industrial.
    \item A integração com DCOM possibilita a comunicação remota entre sistemas, expandindo o alcance do protocolo e permitindo a comunicação entre diferentes plataformas e redes.
\end{itemize}

\subsection{Histórico}
\begin{itemize}
    \item O desenvolvimento do OPC começou nos anos 1990, com a primeira versão, OPC 1.0, lançada em 1996, visando padronizar a comunicação entre sistemas industriais heterogêneos.
    \item No final da década de 1990, a Fundação OPC (OPC Foundation) foi criada para promover a padronização e o desenvolvimento contínuo do protocolo, resultando em versões aprimoradas, como OPC 2.0.
    \item No início dos anos 2000, o OPC/DCOM consolidou sua presença em várias indústrias, como automotiva, farmacêutica e de energia, e a integração com DCOM expandiu as capacidades do protocolo para comunicação remota.
\end{itemize}

\subsection{Funcionamento}
\begin{itemize}
    \item O OPC clássico se divide em três principais especificações: \textbf{OPC DA (Data Access)}, \textbf{OPC A\&E (Alarm and Events)}, e \textbf{OPC HDA (Historical Data Access)}, cada uma voltada para diferentes aspectos da comunicação em sistemas industriais.
    \item O OPC UA (Unified Architecture) é uma evolução do OPC clássico, oferecendo um modelo de informação orientado a objetos e suporte para comunicação robusta e segura através de protocolos como TCP e HTTPS.
    \item O OPC UA suporta dois formatos de comunicação: \textbf{UA Binário} (dados serializados em bytes) para eficiência de transmissão, e \textbf{XML} para integração em níveis mais altos da planta.
\end{itemize}

\subsection{Meio Físico}
\begin{itemize}
    \item O OPC/DCOM utiliza as redes Ethernet e TCP/IP como meio físico de comunicação, aproveitando a infraestrutura de rede existente nas indústrias.
    \item Cabos Ethernet de cobre e fibra óptica são comumente usados, dependendo dos requisitos de velocidade, distância e imunidade a interferências eletromagnéticas.
    \item Switches e roteadores são dispositivos de rede essenciais para gerenciar a comunicação entre diferentes sistemas e garantir a eficiência e confiabilidade da rede OPC/DCOM.
\end{itemize}

\subsection{Empresas Vendedoras}
\begin{itemize}
    \item \textbf{Rockwell Automation}: Líder no fornecimento de soluções de automação industrial, incluindo servidores e clientes OPC para diversas aplicações.
    \item \textbf{Siemens}: Oferece uma ampla gama de produtos e serviços relacionados ao OPC/DCOM, incluindo sistemas de controle, software de monitoramento e soluções de integração.
    \item \textbf{ABB}: Especialista em automação industrial e energia, fornece soluções baseadas em OPC/DCOM para comunicação e controle em várias indústrias.
    \item \textbf{Schneider Electric}: Empresa líder em gerenciamento de energia e automação, oferecendo controladores, softwares de monitoramento e sistemas de integração compatíveis com OPC/DCOM.
\end{itemize}

\subsection{Vantagem}
\begin{itemize}
    \item \textbf{Interoperabilidade}: Permite a comunicação entre sistemas de controle de diferentes fabricantes, promovendo a integração de diversos equipamentos e softwares em um único ambiente.
    \item \textbf{Eficiência}: A padronização do OPC simplifica a comunicação entre sistemas, reduzindo o tempo de configuração e a complexidade da integração, aumentando a eficiência e a produtividade.
    \item \textbf{Confiabilidade}: O OPC UA é robusto e garante a integridade e precisão dos dados transmitidos, crucial para a tomada de decisões em tempo real.
    \item \textbf{Segurança}: Oferece mecanismos de segurança que protegem os dados em trânsito e garantem a autenticação dos usuários e a autorização de acesso.
\end{itemize}

\subsection{Desvantagem}
\begin{itemize}
    \item \textbf{Dependência de Windows}: O OPC clássico é fortemente dependente do sistema operacional Windows, o que pode limitar sua aplicabilidade em outras plataformas.
    \item \textbf{Complexidade de Implementação}: A configuração e o gerenciamento do OPC/DCOM podem ser complexos, exigindo conhecimento técnico especializado.
    \item \textbf{Performance}: Em ambientes com grande volume de dados, a performance do OPC pode ser impactada, exigindo otimização e ajustes na configuração para garantir a eficiência.
    \item \textbf{Segurança}: O OPC clássico possui poucas camadas de segurança, o que pode resultar em vulnerabilidades, especialmente em ambientes distribuídos.
\end{itemize}



\section{AS-i (Actuator Sensor Interface)}

\subsection{Protocolo}
\begin{itemize}
    \item O AS-i (Actuator Sensor Interface) é um protocolo de comunicação de baixo custo e fácil instalação, projetado para conectar sensores e atuadores binários a controladores lógicos programáveis (CLPs) ou sistemas de controle distribuído (DCS).
    \item Seu principal objetivo é substituir a fiação convencional, reduzindo a complexidade e o custo da instalação em sistemas de automação, conectando todos os dispositivos ao controlador através de um único par de fios.
    \item O AS-i se localiza no nível 1 da pirâmide de redes industriais, onde ocorre a comunicação direta entre sensores e atuadores, fundamental para o controle de processos industriais.
\end{itemize}

\subsection{Histórico}
\begin{itemize}
    \item O AS-i foi desenvolvido em 1990 por um consórcio de onze empresas do setor de automação, com o objetivo de estabelecer um padrão global para automação industrial no nível de campo, especificamente na categoria de Sensor Bus.
    \item Em 1998, a rede AS-i foi padronizada pela norma EN 50295, o que contribuiu para sua aceitação e expansão na indústria.
    \item Desde sua criação, o AS-i tem sido amplamente utilizado como uma solução simples e eficaz para interconectar sensores e atuadores, complementando outros sistemas de automação.
\end{itemize}

\subsection{Funcionamento}
\begin{itemize}
    \item O AS-i opera em uma topologia mestre-escravo com polling cíclico. O mestre controla toda a rede, realizando o polling nos escravos e gerenciando a comunicação.
    \item Os dispositivos na rede são denominados escravos e só têm acesso à rede quando requisitados pelo mestre. Cada escravo pode conectar até quatro sensores e quatro atuadores.
    \item A rede suporta comunicação de dados e alimentação de dispositivos utilizando o mesmo par de fios, simplificando a instalação e manutenção.
    \item A transferência de valores analógicos é possível, embora demande quatro ciclos de rede para ser concluída, o que é adequado para aplicações que não exigem altas taxas de transferência de dados.
\end{itemize}

\subsection{Meio Físico}
\begin{itemize}
    \item O meio físico utilizado no AS-i é um cabo específico denominado cabo AS-i, composto por dois condutores sem blindagem. Este cabo é projetado para evitar a inversão de polaridade e transportar tanto dados quanto energia para os dispositivos conectados.
    \item Existem dois tipos principais de cabos AS-i:
    \begin{itemize}
        \item \textbf{Cabo Flat Amarelo:} Possui um formato geométrico que evita a inversão de polaridade ao ser utilizado com conectores tipo vampiro.
        \item \textbf{Cabo Redondo:} Com diâmetro de 8 mm, é adequado para conexões com prensa-cabos de segurança aumentada. Foi desenvolvido para atender às características elétricas exigidas pela rede AS-i, como impedância e capacitância.
    \end{itemize}
    \item O comprimento máximo dos cabos é de 100 metros por segmento, podendo ser expandido através de repetidores para até três segmentos.
\end{itemize}

\subsection{Empresas Vendedoras}
\begin{itemize}
    \item As principais empresas que fornecem produtos compatíveis com o protocolo AS-i incluem Siemens, ABB, Phoenix Contact, Pepperl+Fuchs, e Bihl+Wiedemann.
    \item Esses fornecedores oferecem uma gama de dispositivos como mestres AS-i, escravos, fontes de alimentação específicas, e módulos de I/O, todos otimizados para uso em redes AS-i.
\end{itemize}

\subsection{Vantagem}
\begin{itemize}
    \item \textbf{Simplicidade de Instalação:} O AS-i reduz significativamente a quantidade de cabos necessários, facilitando a conexão de dispositivos e diminuindo o tempo de instalação.
    \item \textbf{Flexibilidade:} Permite fácil expansão e modificação da rede, além de integrar novos dispositivos sem interromper a operação existente.
    \item \textbf{Custo-Efetividade:} A redução da complexidade da fiação e o tempo menor de instalação resultam em economia significativa de custos.
    \item \textbf{Diagnóstico e Manutenção:} Oferece capacidades avançadas de diagnóstico, facilitando o monitoramento e a manutenção do sistema.
\end{itemize}

\subsection{Desvantagem}
\begin{itemize}
    \item \textbf{Largura de Banda Limitada:} O AS-i possui uma largura de banda relativamente baixa, o que pode restringir a quantidade de dados transmitidos em comparação com outras tecnologias de automação mais avançadas.
    \item \textbf{Capacidade Limitada:} Cada segmento da rede AS-i é limitado em termos de comprimento e número de dispositivos, o que pode não ser ideal para grandes instalações industriais que exigem maior escalabilidade.
\end{itemize}

\section{Ethernet/IP}

\subsection{Protocolo}
\begin{itemize}
    \item Ethernet/IP (Ethernet Industrial Protocol) é um protocolo de comunicação industrial que integra a tecnologia Ethernet com o Common Industrial Protocol (CIP), permitindo a comunicação eficiente em redes industriais e corporativas.
    \item É amplamente utilizado para controle, monitoramento e configuração de dispositivos em sistemas de automação industrial, permitindo a comunicação em tempo real e a integração com redes de TI.
    \item Ethernet/IP é baseado nos padrões Ethernet e TCP/IP, o que facilita sua adoção em infraestruturas de rede existentes, suportando uma ampla gama de dispositivos e aplicações.
\end{itemize}

\subsection{Histórico}
\begin{itemize}
    \item Ethernet/IP foi desenvolvido no final dos anos 1990 e lançado oficialmente em 2000 pela ODVA (Open DeviceNet Vendors Association), como uma extensão do CIP, já utilizado em outros protocolos industriais como DeviceNet e ControlNet.
    \item Desde então, o Ethernet/IP tem sido amplamente adotado em diversas indústrias devido à sua capacidade de suportar aplicações críticas em tempo real e sua compatibilidade com as infraestruturas de TI existentes.
    \item A evolução contínua da Ethernet/IP, incluindo melhorias na segurança e no suporte a aplicações de alto desempenho, contribuiu para sua popularidade e ampla adoção global.
\end{itemize}

\subsection{Funcionamento}
\begin{itemize}
    \item Ethernet/IP funciona utilizando o modelo de comunicação Ethernet padrão, combinado com o protocolo CIP para gerenciar a comunicação entre dispositivos industriais.
    \item Suporta tanto comunicação cíclica (em tempo real) quanto acíclica (não crítica), permitindo a transmissão de dados de controle, monitoramento e configuração.
    \item A comunicação em tempo real é habilitada pelo uso de mecanismos como QoS (Quality of Service) e técnicas de priorização de pacotes, garantindo que os dados críticos sejam transmitidos com baixa latência e alta confiabilidade.
    \item Ethernet/IP opera em redes LAN e WAN, facilitando a conectividade entre diferentes níveis de uma planta industrial, desde dispositivos de campo até sistemas de gestão corporativa.
\end{itemize}

\subsection{Meio Físico}
\begin{itemize}
    \item Ethernet/IP utiliza cabos de par trançado (Cat 5e, Cat 6) e conectores RJ45 para comunicação em redes Ethernet padrão, suportando velocidades que variam de 10 Mbps a 1 Gbps.
    \item Em ambientes industriais rigorosos, o uso de conectores industriais M12 e cabos blindados é comum para garantir a durabilidade e a imunidade a interferências eletromagnéticas.
    \item Ethernet/IP também pode ser implementado sobre fibra óptica, oferecendo maior imunidade a interferências e suporte para longas distâncias, tornando-o adequado para grandes instalações industriais.
    \item Switches e roteadores industriais robustos são essenciais para garantir a confiabilidade e o desempenho da rede Ethernet/IP em ambientes industriais exigentes.
\end{itemize}

\subsection{Empresas Vendedoras}
\begin{itemize}
    \item As principais empresas que oferecem produtos compatíveis com Ethernet/IP incluem Rockwell Automation, Siemens, Schneider Electric, Cisco, Omron, e Belden.
    \item Esses fornecedores oferecem uma variedade de dispositivos como CLPs, switches Ethernet industriais, módulos de I/O, e gateways de comunicação, todos otimizados para redes Ethernet/IP.
\end{itemize}

\subsection{Vantagem}
\begin{itemize}
    \item \textbf{Alta Velocidade e Escalabilidade:} Ethernet/IP oferece alta velocidade de comunicação, suportando aplicações em tempo real e é altamente escalável, desde pequenas redes até grandes instalações industriais.
    \item \textbf{Integração com TI:} Sua compatibilidade com padrões Ethernet e TCP/IP facilita a integração de sistemas de automação industrial com redes de TI corporativas, permitindo o uso de ferramentas de TI padrão para gestão e monitoramento.
    \item \textbf{Interoperabilidade:} Ethernet/IP é amplamente suportado por uma variedade de dispositivos de diferentes fabricantes, garantindo a interoperabilidade em sistemas heterogêneos.
\end{itemize}

\subsection{Desvantagem}
\begin{itemize}
    \item \textbf{Complexidade de Configuração:} A configuração de redes Ethernet/IP pode ser complexa, exigindo conhecimento especializado para garantir o desempenho ideal e a segurança da rede.
    \item \textbf{Necessidade de Infraestrutura Robusta:} Para suportar as demandas de aplicações industriais críticas, é necessário uma infraestrutura de rede robusta, incluindo switches e roteadores industriais de alta qualidade, o que pode aumentar os custos de implementação.
    \item \textbf{Segurança:} Ethernet/IP, como qualquer protocolo baseado em Ethernet, pode ser vulnerável a ataques cibernéticos se não forem implementadas medidas de segurança adequadas, como firewalls e criptografia.
\end{itemize}

\end{document}

\section{Conclusão}
\begin{itemize}
    \item \textbf{Resumo}: Resumo das principais diferenças e semelhanças entre os protocolos apresentados.
    \item \textbf{Considerações Finais}: Reflexão sobre a importância da escolha do protocolo correto para cada aplicação industrial.
    \item \textbf{Perspectivas Futuras}: Breve discussão sobre possíveis evoluções ou tendências em protocolos de redes industriais.
\end{itemize}

\end{document}